\documentclass[a4paper,12pt]{article}
\usepackage{fontspec}
\usepackage{polyglossia}
\setmainlanguage{russian}
\setotherlanguage{english}
\setmainfont{DejaVu Serif}
\setsansfont{DejaVu Sans}
\setmonofont{DejaVu Sans Mono}

\usepackage{geometry}
\geometry{left=2cm,right=2cm,top=2cm,bottom=2cm}

\usepackage{listings}
\usepackage{xcolor}
\usepackage{hyperref}
\usepackage{enumitem}
\usepackage{graphicx}

\definecolor{codegreen}{rgb}{0,0.6,0}
\definecolor{codegray}{rgb}{0.5,0.5,0.5}
\definecolor{codepurple}{rgb}{0.58,0,0.82}
\definecolor{backcolour}{rgb}{0.95,0.95,0.92}

\lstdefinestyle{mystyle}{
    backgroundcolor=\color{backcolour},
    commentstyle=\color{codegreen},
    keywordstyle=\color{magenta},
    numberstyle=\tiny\color{codegray},
    stringstyle=\color{codepurple},
    basicstyle=\ttfamily\footnotesize,
    breakatwhitespace=false,
    breaklines=true,
    captionpos=b,
    keepspaces=true,
    numbers=left,
    numbersep=5pt,
    showspaces=false,
    showstringspaces=false,
    showtabs=false,
    tabsize=2,
    frame=single
}

\lstset{style=mystyle}

\title{Документация к системе управления студенческой базой данных}
\date{Версия 1.0 \\ \today}

\begin{document}

\maketitle

\section*{Краткое описание}
Система представляет собой клиент-серверное приложение для управления базой данных студентов с возможностью:
\begin{itemize}[noitemsep]
\item Добавления, удаления и изменения записей
\item Поиска по различным критериям
\item Сохранения и загрузки данных из файла
\item Сетевого взаимодействия между клиентом и сервером
\end{itemize}

\tableofcontents

\section{Требования к системе}
\begin{itemize}
\item Linux-система (рекомендуется Ubuntu 20.04+)
\item g++ версии 9.0+
\item Утилита make
\item Пакет texlive-full для генерации документации
\item Доступ к порту 8081 для сетевого взаимодействия
\end{itemize}

\section{Установка и запуск}
\subsection{Установка зависимостей}
\begin{lstlisting}[language=bash]
sudo apt update
sudo apt install g++ make texlive-full texlive-lang-cyrillic
\end{lstlisting}

\subsection{Сборка проекта}
\begin{enumerate}
\item Клонируйте репозиторий или скопируйте файлы в рабочую директорию
\item Выполните в терминале:
\begin{lstlisting}[language=bash]
make all
\end{lstlisting}
\end{enumerate}

\subsection{Запуск системы}
\begin{enumerate}
\item Запустите сервер в отдельном терминале:
\begin{lstlisting}[language=bash]
./server
\end{lstlisting}

\item Запустите клиент в другом терминале:
\begin{lstlisting}[language=bash]
./client
\end{lstlisting}

\item Для запуска с тестовым сценарием:
\begin{lstlisting}[language=bash]
./client test_scenario.txt
\end{lstlisting}
\end{enumerate}

\section{Формат данных}
\subsection{Структура файла данных}
Данные хранятся в CSV-файле \texttt{students.csv}:
\begin{lstlisting}
Иван Иванов,101,4.5,Примечание1
Петр Петров,102,3.8,Примечание2
\end{lstlisting}

\subsection{Структура студента}
\begin{lstlisting}[language=C++]
struct Student {
    std::string name;    // ФИО студента
    int group;           // Номер группы
    double rating;       // Рейтинг (2.0-5.0)
    std::string info;    // Дополнительная информация
};
\end{lstlisting}

\section{Руководство пользователя}
\subsection{Команды клиента}
\begin{tabular}{|l|p{8cm}|}
\hline
\textbf{Команда} & \textbf{Описание} \\ \hline
GET\_ALL & Получить список всех студентов \\ \hline
SELECT \textit{условие} & Фильтрация (пример: \texttt{SELECT rating>4.0;group=101}) \\ \hline
RESELECT \textit{условие} & Уточнение предыдущего выбора \\ \hline
ADD|\textit{имя}|\textit{группа}|\textit{рейтинг}|\textit{инфо} & Добавление студента \\ \hline
DELETE \textit{имя} & Удаление студента \\ \hline
FORMAT [html|text] [файл] & Установка формата вывода \\ \hline
HELP & Вывод справки \\ \hline
EXIT & Выход из программы \\ \hline
\end{tabular}

\subsection{Примеры использования}
\begin{lstlisting}[language=bash]
# Добавление студента
ADD|Сидоров Алексей|103|4.2|Активен в науке

# Поиск студентов с рейтингом >4.0 из группы 101
SELECT rating>4.0;group=101

# Сохранение результатов в HTML
FORMAT html result.html
SELECT rating>3.5
\end{lstlisting}

\section{Архитектура системы}
\subsection{Серверная часть}
\begin{itemize}
\item Многопоточная обработка подключений
\item Использование мьютексов для синхронизации
\item Порт по умолчанию: 8081
\end{itemize}

\subsection{Клиентская часть}
\begin{itemize}
\item Поддержка интерактивного режима
\item Возможность выполнения сценариев
\item Два формата вывода: текстовый и HTML
\end{itemize}

\section{Генерация документации}
Для генерации PDF-документации выполните:
\begin{lstlisting}[language=bash]
make documentation.pdf
\end{lstlisting}

Или вручную:
\begin{lstlisting}[language=bash]
xelatex documentation.tex
\end{lstlisting}

\section{Тестирование}
Включен тестовый сценарий \texttt{test\_scenario.txt}:
\begin{lstlisting}
GET_ALL
ADD|Тестовый Студент|999|5.0|Тестовая запись
SELECT group=999
DELETE Тестовый Студент
\end{lstlisting}


\end{document}
